\chapter{htt Script Structure}
\label{chap:structure}
\begin{flushleft}
A htt test usually consists of a ".htt" file (the main script), and optionally
some include files (usually with the suffix ".htb").

The main script consists of \textit{include statements}, global or local 
\textit{commands}, \textit{variables}, \textit{bodies} and comments:

\begin{itemize}
  \item \textit{Command}: An instruction to perform an action, like sending a request,
          setting a header in that request, evaluating the response etc. There are 
          global and local commands (explained in the chapter "Commands").
  \item \textit{Include}: A global command used to include re-usable scripting code from other files.
  \item \textit{Variables}: Holder for values created during the execution of the script. 
  \item \textit{Body}: A block of code containing a number of commands.
  \item \textit{Comment}: A line of non-executable text
\end{itemize}

Commands, variables and body types are explained in more detail in the corresponding chapters. 

\subsection{Comments}

Comment lines must begin with the "\char`\#" character.

\newpage
\section{Structure Example}

A simple htt script code example for showing the structure:
\end{flushleft}

\begin{usplisting}
    # Include some macros
    INCLUDE macros.htb
    
    # The actual test client part
    CLIENT
       ...
    END
    
    # Optional: a re-used code block
    BLOCK DOIT
       ...
    END
    
    # Optional: A test monitoring daemon
    DAEMON
       ...
    END 
\end{usplisting}

In the example above, \texttt{INCLUDE} is a global statement used to include another 
script file with some re-usable code blocks. It is followed by the \texttt{CLIENT} 
block which starts the actual HTTP test client, a block of re-usable functionality 
named "DOIT" and a \texttt{DAEMON} body which contains some test monitoring functionality.

\section{Control Flow}

The following options are available to control the flow of a httest script:

\begin{itemize}
\item{Conditions}: The \texttt{\_IF} command supports the execution of a code body 
                   depending on a condition.
\item{Loops}: While the \texttt{\_FOR} command allows to execute a code body for a given
             list of elements, the \texttt{\_LOOP} command allows to do the same for a 
             certain number of times.
\end{itemize}