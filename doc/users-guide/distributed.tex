\chapter{Distribute Clients and Servers}
\label{chap:distributed}

httest 2.2.12 and higher can distribute clients and servers to remote hosts. 

Distribute clients will be typically used for performance and load tests. Where
distribute servers typically will be used for sophisticated integration tests.


\section{Distribute Clients}
\label{chap:distributeClients}

If you like to run loadtests and your maschine is not fast enough, you could
increase the load with additional maschines. Till today you have to do this manually
by copy past the test script to different maschines and run it seperatly.
Today it is possible to do this in one single script.

With the command

\begin{usplisting}
  PERF:DISTRIBUTED <host>:<port>
\end{usplisting}

you can add additional hosts where your clients will be distributed. The local host 
is automaticaly included. The clients are distributed with round robbin starting with
your local host. If a remote host is not accessable it will be skipped.

Now you need clients to distribute normaly done this way

\begin{usplisting}
  CLIENT <n>
  <body>
  END
\end{usplisting}

see also in the global command section.

Of course you can have many differen client as well.

With htremote a remote acceptor for the serialized httest clients must be started.Could
be done even in your httest script. At the moment htremote do not have a deamon mode but
will comming soon.

\begin{usplisting}
  htremote -p <port> -e "httest -Ss"
\end{usplisting}

The option of httest -S do start httest in shell mode to feed the script over standard in.
The second option -s do make the httest silent, for debug purpose you could avoid the second
option.

\section{Distribute Servers}
\label{chap:distributeClients}

If your integration test case needs a mock on a remote host this you will use this feature.
You just have to define your server this way

\begin{usplisting}
  SERVER [SSL:]<port> [<n>] -> <remote-host>:<remote-port>
  <body>
  END
\end{usplisting}

see also int the global command section.

This server will now be serialzied to remote-host:remote-port.

With htremote a remote acceptor for the serialized httest servers must be started.Could
be done even in your httest script. At the moment htremote do not have a deamon mode but
will comming soon.

\begin{usplisting}
  htremote -p <port> -e "httest -S"
\end{usplisting}

The option of httest -S do start httest in shell mode to feed the script over standard in.

