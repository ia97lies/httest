% =========================================
% COMMAND: _SSL_CONNECT
% =========================================

\newpage
\section{\_SSL\_ACCEPT}
\label{cmd:_SSL_ACCEPT}

\paragraph{Syntax:}
\subparagraph{}
\texttt{\_SSL\_ACCEPT SSL|SSL2|SSL3|TLS1 [<cert-file> <key-file> [ca-cert-file]]}

\paragraph{Purpose:}
\subparagraph{}
Performs an SSL accept on an already accepted TCP socket. This is usefule for protocols where
the SSL handshake is done after some plain text exchange, for example POP3 TLS server.

\paragraph{Parameters:}
\subparagraph{}
\textit{\texttt{SSL}}: Defines the type of SSL connection. Must be one of these values: 
\texttt{SSL}, \texttt{SSL2}, \texttt{SSL3} or \texttt{TLS1}.

\subparagraph{}
\textit{\texttt{cert-file}}, \textit{\texttt{key-file}}, \textit{\texttt{ca-cert-file}}: Optional; required 
for SSL connections. 

