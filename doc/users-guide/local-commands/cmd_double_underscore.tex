% =========================================
% COMMAND: __
% =========================================

\newpage
\section{\_\_}
\label{cmd:__}

\paragraph{Syntax:}
\subparagraph{}
\texttt{\_\_<string>}

\paragraph{Purpose:}
\subparagraph{}
Sends the given string data to the current socket, with  
CRLF line termination characters.

\subparagraph{}
NOTE: There MUST NOT BE a blank between the command and the 
following string data (or the blank itself will be part of the 
string data as well).

\subparagraph{}
The reason for using this peculiar design lies in the history 
of the httest utility. In early versions, the value of the "content-length" 
HTTP header of a request had to be calculated and set manually. In that 
situation, it $<$n$>$ was easier to calculate the actual number of bytes 
a line of data would take up, because the two underscores at the 
beginning of the line allowed to just use the number of columns 
value displayed by any text editor. The reason: Without 
these two characters, a text editor would display only the number 
of visible characters, but the CRLF characters had also to be taken 
into account for a correct "content-length" header value. So the 
double underscore was a replacement for the invisible line termination 
characters.

\subparagraph{}
However, it's not necessary for this purpose anymore as httest 
allows to set the correct content length value automatically now.