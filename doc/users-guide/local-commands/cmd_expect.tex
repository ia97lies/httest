% =========================================
% COMMAND: _EXPECT
% =========================================

\newpage
\section{\_EXPECT}
\label{cmd:_EXPECT}

\paragraph{Syntax:}
\subparagraph{}
\texttt{\_EXPECT .|headers|body|error|exec|var() "|'[!]<regex>"|'}

\paragraph{Purpose:}
\subparagraph{}
Defines what data is expected to be received, either on the current connection, 
or the stdout of an executed program, or in a certain variable. 
A regular expression is used to define the expected data.

\subparagraph{}
Use a preceeding exclamation mark ("!") before the regular expression 
to negate the check.

\subparagraph{}
NOTE: This command does not perform any check yet. The actual check 
on the data is performed in the following \texttt{\_WAIT} command. 

\subparagraph{}
Example 1: Expect a certain output from an executed binary:

\begin{usplisting}
    _EXPECT exec "myValue"
    _EXEC echo myValue
\end{usplisting}

\subparagraph{}
Example 2: Expect a certain value in variable \$A:

\begin{usplisting}
    _SET A=myValue
    _EXPECT var(A) "myValue"
\end{usplisting}

