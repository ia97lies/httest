\section{Forward Proxy}

Using a plain forward proxy in your test is very straight forward. The following
example will show it.
\begin{usplisting}
    CLIENT
      _REQ my.forward.proxy 6667
      __GET http://www.wikipedia.com/ HTTP/1.1
      __Host: www.wikipedia.com
      __User-Agent: httest
      __
      _WAIT
    END
\end{usplisting}

The GET line do contain the full qualified URL including scheme, hostname and 
optional the port.

\section{Forward Proxy with SSL}

Using a SSL forward proxy in your test is a more complicated task. You have to
tell the forward proxy first where to connect and the SSL connection must be
done after success.
\begin{usplisting}
    CLIENT
    _REQ my.forward.proxy 6667
    __CONNECT https://www.wikipedia.com
    __
    _EXPECT headers "HTTP/1.1 200"
    _WAIT
    _SSL:CONNECT SSL
    __GET / HTTP/1.1
    __Host: www.wikipedia.com
    __
    _WAIT
    END
\end{usplisting}

With \texttt{\_SSL:CONNECT SSL} the SSL handshake is done after TCP connect.
