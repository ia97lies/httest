\chapter{Quickstart}
\label{chap:quickstart}

This short chapter shows some simple, basic htt examples without explaining 
everything. Some parts should be obvious for anyone who has a solid knowledge 
of HTTP, everything else will be explained in more detail in the other chapters.

NOTE: Depending on the version and platform you are using \texttt{httest} on, there may be 
a bug which requires that every script ends with an empty line, below the last
statement or variable definition!

\section{Step 1: GET www.wikipedia.org} 

This first example sends a simple GET request to "www.wikipedia.org" and checks 
that the page contains the word "Wikipedia", and the HTTP response code is 200.

Enter the following example code into a file named "wikipedia.htt":

% ---------------------- SAMPLE BOX ----- BEGIN ------------
\begin{usplisting}
    CLIENT
    _REQ www.wikipedia.org 80
    __GET / HTTP/1.1
    __Host: www.wikipedia.org
    __
    _EXPECT . "200 OK"
    _EXPECT . "Wikipedia"
    _WAIT
    END
\end{usplisting}
% ---------------------- SAMPLE BOX ----- END ------------

Then execute the newly created script using the \texttt{httest} binary:

% ---------------------- SAMPLE BOX ----- BEGIN ------------
\begin{usplisting}
    %> httest wikipedia.htt
\end{usplisting}
% ---------------------- SAMPLE BOX ----- END ------------

\newpage 
And, given that the system you are running \texttt{httest} on has internet 
access, the following result should be displayed:

% ---------------------- SAMPLE BOX ----- BEGIN ------------
\begin{usplisting}
    run wikipedia.htt
    CLT0-0 start ...
    _REQ www.wikipedia.org 80
    __GET / HTTP/1.1
    __Host: www.wikipedia.org
    __
    _EXPECT . "200 OK"
    _EXPECT . "Wikipedia"
    _WAIT
    >GET / HTTP/1.1
    >Host: www.wikipedia.org
    >
    <HTTP/1.0 200 OK
    <Date: Tue, 18 Jan 2011 12:38:51 GMT
    ...
    ...many more headers, and the HTML page...
    ...
    <</body>
    <</html> OK
\end{usplisting}
% ---------------------- SAMPLE BOX ----- END ------------

Some explanations concerning the output:

\begin{itemize}
\item The first part (everything before the lines with the "$>$" prefix)
      shows the commands executed by the httest binary.
\item All lines beginning with "$>$" show the data sent to the HTTP server.
\item All lines beginning with "$<$" show the data received in the HTTP response.
\item The "OK" at the end of the output and the exit code 0 of httest signals a successful test completion.
\end{itemize}

\newpage
\section{Step 2: Externalize host and port values} 

Create a new file named "\texttt{values.htb}" in the same directory like the 
"\texttt{wikipedia.htt}" file. Insert these contents:

% ---------------------- SAMPLE BOX ----- BEGIN ------------
\begin{usplisting}
    # Set host and port as variables
    SET PORT=80
    SET HOST=www.wikipedia.org
\end{usplisting}
% ---------------------- SAMPLE BOX ----- END ------------

Then change the contents of the "\texttt{wikipedia.htt}" file like this:

% ---------------------- SAMPLE BOX ----- BEGIN ------------
\begin{usplisting}
    INCLUDE values.htb
    
    CLIENT
    _REQ $HOST $PORT
    __GET / HTTP/1.1
    __Host: www.wikipedia.org
    __
    _EXPECT . "200 OK"
    _EXPECT . "Wikipedia"
    _WAIT
    END
\end{usplisting}
% ---------------------- SAMPLE BOX ----- END ------------

Now you can use that include file to use the same host and port value 
in multiple htt scripts, while being able to maintain those values in 
only one file.

